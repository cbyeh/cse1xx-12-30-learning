\documentclass[12pt, oneside]{article}
\usepackage[letterpaper,scale=0.85, centering]{geometry}
\usepackage{amssymb,amsmath}
\usepackage{currfile,xstring,hyperref}
\usepackage[labelformat=empty]{caption}
\usepackage[dvipsnames,table]{xcolor}
\usepackage{multicol}
\usepackage{listings}

% NOTE(joe): This environment is credit @pnpo (https://tex.stackexchange.com/a/218450)
\lstnewenvironment{algorithm}[1][] %defines the algorithm listing environment
{   
    \lstset{ %this is the stype
        mathescape=true,
        frame=tB,
        numbers=left, 
        numberstyle=\tiny,
        basicstyle=\rmfamily\scriptsize, 
        keywordstyle=\color{black}\bfseries,
        keywords={,procedure, div, for, to, input, output, return, datatype, function, in, if, else, foreach, while, begin, end, }
        numbers=left,
        xleftmargin=.04\textwidth,
        #1
    }
}
{}

\lstnewenvironment{java}[1][]
{   
    \lstset{
        language=java,
        mathescape=true,
        frame=tB,
        numbers=left, 
        numberstyle=\tiny,
        basicstyle=\ttfamily\scriptsize, 
        keywordstyle=\color{black}\bfseries,
        keywords={, int, double, for, return, if, else, while, }
        numbers=left,
        xleftmargin=.04\textwidth,
        #1
    }
}
{}

\setlength{\parindent}{0em}
\setlength{\parskip}{0.5em}

\title{\bf CSE 103 \\[2ex]
       \Large Homework \#1\\ Fall 2019}
\begin{document}
\date{\textbf{Due}: Tue, October 8, 2019 at 11:00PM on Gradescope}
\maketitle
%$\\[-50pt]$

\section{Directions}
There are two goals to this first assignment: (1) To practice
typesetting your answers and submitting the HW through gradescope. (2)
To do a quick review of basic concepts of combinatorics and probability
that were covered in CSE21.

This assignment will be submitted individually and is \textbf{open to
collaboration with other students}.  You can post public questions about the
assignment to Piazza, discuss the questions and their answers with other
students, and ask questions in office hours. 

Your answers have to be typeset, not handwritten. This is for two
reasons: (a) to reduce ambiguity of the answers, and (b) to be kind to
the TA's eyesight. We recommend you use latex, but you can also use
word-processors that support mathematical formulas. More directions
are available here: {\tt https://tinyurl.com/y2gv9bn9}. A latex
version of the HW is available from the class plan here: {\tt https://sites.google.com/eng.ucsd.edu/cse103/class-plan}.

You will submit this assignment via Gradescope
(\url{https://www.gradescope.com}) in the assignment called ``Homework
1''. You can submit each question as many times as you like. You should solve the problems and ask
questions about them offline first, then try submitting once you are confident
in your answers. \textbf{No late submissions are accepted.}

\noindent {\large{\textbf{Note:} Please enter the solution for
    each problem after the question and use {\bf Gradescope} to mark
    a rectangle around your answer.}}

Special Latex Notation useful in this assignment: m choose k: ${m
  \choose k}$
\section{Problems}
\begin{enumerate}

\item Find the number of different ways of arranging two {\tt R}'s and
  one {\tt G}'s in a row. Write out all the patterns.
\item Find the number of different ways of arranging two {\tt R}'s and
  three {\tt G}'s in a row. Write out all the patterns.
\item A box contains two red balls and three green ones. Four draws
  are made at random with replacement from the box. Find the chance
  that:\\
  (write the exact expression and the computed probability as
  integer percetages)
\begin{enumerate}
    \item a red ball is never drawn
    \item a red ball appears exactly once
    \item a red ball appears exactly twice
    \item a red ball appears exactly three times
    \item a red ball appears on all the draws
    \item a red ball appears at least twice
\end{enumerate}
\item A die is rolled five times. Find the chance that:
\begin{enumerate}
    \item an ace (one dot) never appears
    \item an ace appears exactly twice
    \item an ace appears exactly five times
\end{enumerate}

\end{enumerate}
\end{document}

